\documentclass[a4paper]{article}
\usepackage{hyperref}
\usepackage{xcolor}
\usepackage{graphicx}
\usepackage[english]{babel}
\usepackage[T1]{fontenc}
\usepackage{url}
\usepackage{color}
\usepackage{fancyhdr}
\usepackage{amssymb}
\usepackage{amsmath} 
\usepackage[margin=2.5cm]{geometry} 
\usepackage{listings}
\usepackage[utf8x]{inputenc}
%\date{}
\usepackage{textcomp}

\definecolor{comment}{rgb}{0,128,0} % dark green
\definecolor{string}{rgb}{255,0,0}  % red
\definecolor{keyword}{rgb}{0,0,255} % blue

\lstdefinestyle{customc}{
    backgroundcolor=\color{white},   
    commentstyle=\color{comment},
    keywordstyle=\color{blue},
    numberstyle=\tiny\color{gray},
    stringstyle=\color{red},
    basicstyle=\footnotesize,
    breakatwhitespace=false,    
    frame = single,
    framexleftmargin=15pt,     
    breaklines=true,                 
    captionpos=b,                    
    keepspaces=true,                 
    numbers=left,                    
    numbersep=5pt,                  
    showspaces=false,                
    showstringspaces=false,
    showtabs=false,                  
    tabsize=2,
    language=C
}

\begin{document}
 

\title{\vspace{2cm}Relazione sul Progetto dell'Esame di\\ \textbf{Sistemi Operativi}\\ Anno Accademico 2016/17}

\author{Bacciarini Yuri - \texttt{xxxxxxxx} - \href{mailto:yuri.bacciarini@stud.unifi.it}{\textit{yuri.bacciarini@stud.unifi.it}}
   \and Bindi Giovanni - \texttt{5530804} - \href{mailto:giovanni.bindi@stud.unifi.it}{\textit{giovanni.bindi@stud.unifi.it}}
   \and Puliti Gabriele- \texttt{5300140} - \href{mailto:gabriele.puliti@stud.unifi.it}{\textit{gabriele.puliti@stud.unifi.it}}} 



\maketitle

\tableofcontents

\newpage
\section{Primo Esercizio}
\textbf{Simulatore di chiamate a procedura}
\subsection{Descrizione dell'implementazione}
L'obiettivo del primo esercizio \'e quello di implementare uno scheduler di processi. Quest'ultimo deve permettere all'utente di poter creare, eseguire ed eliminare i processi stessi secondo una politica di priorit\'a od esecuzioni rimanenti. \\
Abbiamo organizzato il codice in tre files: due librerie \textit{config.h} e \textit{taskmanager.h} ed un programma, \textit{scheduler.c}. All'interno di \textit{config.h}\ref{config} vengono unicamente definite due stringhe utilizzate nella formattazione dell'output. All'interno di \textit{taskmanager.h}\ref{task} abbiamo invece definito la \texttt{struct} \textit{TaskElement}, ovvero l'elemento \textbf{\textit{Task}}, descritto da 5 campi fondamentali che rappresentano un processo all'interno della nostra implementazione:

\begin{enumerate}
\item \textit{ID} : Un numero intero univoco che viene automaticamente assegnato alla creazione del task.
\item \textit{nameTask} : Nome del task, di massimo 8 caratteri, scelto dall'utente alla creazione.
\item \textit{priority} : Numero intero che rappresenta la priorit\'a del task.
\item \textit{remainingExe} : Numero intero che rappresenta il numero di esecuzioni rimanenti (burst) del task.
\item \textit{*nextTask} : Puntatore al task successivo
\end{enumerate}

Sempre all'interno di \textit{taskmanager.h}\ref{task} vi sono le implementazioni delle operazioni che il nostro scheduler sar\'a in grado di effettuare, definite dalle seguenti funzioni:

\begin{itemize}
\item \texttt{setExeNumber(void)} : Permette l'inserimento del numero di esecuzioni rimanenti $n$, effettuando i controlli sulla legalit\'a dell'input ( $1<n<99$ ).

\item \texttt{setPriority(void)} : Permette l'inserimento della priorit\'a $p$, effettuando i controlli sulla legalit\'a dell'input ( $ 1<p<9$ ).

\item \texttt{setTaskName(Task*)} : Permette l'inserimento del nome del task, effettuando i controlli sulla lunghezza massima della stringa inserita (al massimo 8 caratteri).

\item \texttt{isEmptyTaskList(Task*)} : Esegue il controllo sulla lista di task, restituendo 0 nel caso sia vuota.

\item \texttt{selectTask(Task*)} : Restituisce il task con il PID richiesto dall'utente, dopo aver eseguito la ricerca nella lista.

\item \texttt{modifyPriority(Task*)} : Permette di modificare la priorit\'a del task selezionato. 

\item \texttt{modifyExecNumb(Task*)} : Permette di modificare il numero di esecuzioni rimanenti del task selezionato.

\item \texttt{newTaskElement(Task*,int)} : Permette la creazione di un nuovo task, allocandolo in memoria con l'utilizzo di una \texttt{malloc}.

\item \texttt{printTask(Task*)} : Esegue la stampa degli elementi del task coerentemente con la richiesta nella specifica dell'esercizio.

\item \texttt{printListTask(Task*)} : Esegue la stampa dell'intera lista dei task, richiamando la funzione \texttt{printTask}.

\item \texttt{deleteTask(Task*, Task*)} : Permette l'eliminazione di un task dalla lista, semplicemente collegando il puntatore \textit{nextTask} dell'elemento precedente al task successivo a quello che deve essere eliminato.

\item \texttt{executeTask(Task*)} : Esegue il task in testa alla coda, eseguendo i controlli sul numero di esecuzioni rimanenti.
\end{itemize}

Le operazioni legate allo scheduling sono state poi affidate a \textit{scheduler.c}\ref{sched}, il quale contiene le funzioni:

\begin{itemize}
\item \texttt{getChoice(void)} : Stampa il menu di scelta delle operazioni eseguibili e restituisce la risposta data in input dall'utente.

\item \texttt{switchPolicy(char)} : Permette di modificare la politica di scheduling, passando da priorit\'a ad esecuzioni rimanenti.

\item \texttt{sortListByPriority(Task*)} : Ordina la lista dei task per valori decrescenti della priorit\'a ( $max(p)=9$ ).

\item \texttt{sortListByExecution(Task*)} : Ordina la lista dei task per valori decrescenti del numero di esecuzioni rimanenti ( $max(n)=99$ ).

\item \texttt{swapTask(Task*, Task*, Task*)} : Permette l'inversione dell'ordine di due task.

\item \texttt{main()} : Main del programma.
\end{itemize}

\subsection{Evidenza del corretto funzionamento}
Qu\'i andranno gli screenshot
\subsection{Codice}
\lstinputlisting[label=config, caption=Config, style=customc]{sched/config.h}
\lstinputlisting[label=task, caption=Task Manager, style=customc]{sched/taskmanager.h}
\lstinputlisting[label = sched, caption=Scheduler, style=customc]{sched/scheduler.c}

\newpage
\section{Secondo Esercizio}
\textbf{Esecutore di comandi}
\subsection{Descrizione dell'implementazione}
L'obiettivo del secondo esercizo \'e quello di creare un esecutore di comandi \texttt{UNIX} che scriva, sequenzialmente o parallelamente, l'output dell'esecuzione su di un file. \\
Tutte le funzionalit\'a del programma sono incluse all'interno della libreria \textit{cmd.h}\ref{cmd} e fanno uso a loro volta della libreria \textit{unistd.h}.
La funzione \texttt{initDataFolder()} si occupa di creare la cartella ed inserirvi il file di output. Essa viene generata all'interno della directory \textit{"../commandexe/data/[pid]"} dove il \textit{pid} \'e il process ID del chiamante in questione, ritornato dal \texttt{getpid()}.
Il comando inserito dall'utente viene poi eseguito attraverso una \texttt{popen()}, la quale apre uno stream di scrittura/lettura su di una pipe, inserendovi l'output del comando.
\subsection{Evidenza del corretto funzionamento}
Qu\'i andranno gli screenshot
\subsection{Codice}
\lstinputlisting[label=cmd, caption=Executor, style=customc]{exe/cmd.h}

\newpage
\section{Terzo Esercizio}
\textbf{Message passing}
\subsection{Descrizione dell'implementazione}
\subsection{Evidenza del corretto funzionamento}
Qu\'i andranno gli screenshot
\subsection{Codice}





\end{document}